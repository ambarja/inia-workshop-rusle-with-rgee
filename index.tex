% Options for packages loaded elsewhere
\PassOptionsToPackage{unicode}{hyperref}
\PassOptionsToPackage{hyphens}{url}
\PassOptionsToPackage{dvipsnames,svgnames,x11names}{xcolor}
%
\documentclass[
  letterpaper,
  DIV=11,
  numbers=noendperiod]{scrreprt}

\usepackage{amsmath,amssymb}
\usepackage{iftex}
\ifPDFTeX
  \usepackage[T1]{fontenc}
  \usepackage[utf8]{inputenc}
  \usepackage{textcomp} % provide euro and other symbols
\else % if luatex or xetex
  \usepackage{unicode-math}
  \defaultfontfeatures{Scale=MatchLowercase}
  \defaultfontfeatures[\rmfamily]{Ligatures=TeX,Scale=1}
\fi
\usepackage{lmodern}
\ifPDFTeX\else  
    % xetex/luatex font selection
\fi
% Use upquote if available, for straight quotes in verbatim environments
\IfFileExists{upquote.sty}{\usepackage{upquote}}{}
\IfFileExists{microtype.sty}{% use microtype if available
  \usepackage[]{microtype}
  \UseMicrotypeSet[protrusion]{basicmath} % disable protrusion for tt fonts
}{}
\makeatletter
\@ifundefined{KOMAClassName}{% if non-KOMA class
  \IfFileExists{parskip.sty}{%
    \usepackage{parskip}
  }{% else
    \setlength{\parindent}{0pt}
    \setlength{\parskip}{6pt plus 2pt minus 1pt}}
}{% if KOMA class
  \KOMAoptions{parskip=half}}
\makeatother
\usepackage{xcolor}
\setlength{\emergencystretch}{3em} % prevent overfull lines
\setcounter{secnumdepth}{5}
% Make \paragraph and \subparagraph free-standing
\ifx\paragraph\undefined\else
  \let\oldparagraph\paragraph
  \renewcommand{\paragraph}[1]{\oldparagraph{#1}\mbox{}}
\fi
\ifx\subparagraph\undefined\else
  \let\oldsubparagraph\subparagraph
  \renewcommand{\subparagraph}[1]{\oldsubparagraph{#1}\mbox{}}
\fi

\usepackage{color}
\usepackage{fancyvrb}
\newcommand{\VerbBar}{|}
\newcommand{\VERB}{\Verb[commandchars=\\\{\}]}
\DefineVerbatimEnvironment{Highlighting}{Verbatim}{commandchars=\\\{\}}
% Add ',fontsize=\small' for more characters per line
\usepackage{framed}
\definecolor{shadecolor}{RGB}{241,243,245}
\newenvironment{Shaded}{\begin{snugshade}}{\end{snugshade}}
\newcommand{\AlertTok}[1]{\textcolor[rgb]{0.68,0.00,0.00}{#1}}
\newcommand{\AnnotationTok}[1]{\textcolor[rgb]{0.37,0.37,0.37}{#1}}
\newcommand{\AttributeTok}[1]{\textcolor[rgb]{0.40,0.45,0.13}{#1}}
\newcommand{\BaseNTok}[1]{\textcolor[rgb]{0.68,0.00,0.00}{#1}}
\newcommand{\BuiltInTok}[1]{\textcolor[rgb]{0.00,0.23,0.31}{#1}}
\newcommand{\CharTok}[1]{\textcolor[rgb]{0.13,0.47,0.30}{#1}}
\newcommand{\CommentTok}[1]{\textcolor[rgb]{0.37,0.37,0.37}{#1}}
\newcommand{\CommentVarTok}[1]{\textcolor[rgb]{0.37,0.37,0.37}{\textit{#1}}}
\newcommand{\ConstantTok}[1]{\textcolor[rgb]{0.56,0.35,0.01}{#1}}
\newcommand{\ControlFlowTok}[1]{\textcolor[rgb]{0.00,0.23,0.31}{#1}}
\newcommand{\DataTypeTok}[1]{\textcolor[rgb]{0.68,0.00,0.00}{#1}}
\newcommand{\DecValTok}[1]{\textcolor[rgb]{0.68,0.00,0.00}{#1}}
\newcommand{\DocumentationTok}[1]{\textcolor[rgb]{0.37,0.37,0.37}{\textit{#1}}}
\newcommand{\ErrorTok}[1]{\textcolor[rgb]{0.68,0.00,0.00}{#1}}
\newcommand{\ExtensionTok}[1]{\textcolor[rgb]{0.00,0.23,0.31}{#1}}
\newcommand{\FloatTok}[1]{\textcolor[rgb]{0.68,0.00,0.00}{#1}}
\newcommand{\FunctionTok}[1]{\textcolor[rgb]{0.28,0.35,0.67}{#1}}
\newcommand{\ImportTok}[1]{\textcolor[rgb]{0.00,0.46,0.62}{#1}}
\newcommand{\InformationTok}[1]{\textcolor[rgb]{0.37,0.37,0.37}{#1}}
\newcommand{\KeywordTok}[1]{\textcolor[rgb]{0.00,0.23,0.31}{#1}}
\newcommand{\NormalTok}[1]{\textcolor[rgb]{0.00,0.23,0.31}{#1}}
\newcommand{\OperatorTok}[1]{\textcolor[rgb]{0.37,0.37,0.37}{#1}}
\newcommand{\OtherTok}[1]{\textcolor[rgb]{0.00,0.23,0.31}{#1}}
\newcommand{\PreprocessorTok}[1]{\textcolor[rgb]{0.68,0.00,0.00}{#1}}
\newcommand{\RegionMarkerTok}[1]{\textcolor[rgb]{0.00,0.23,0.31}{#1}}
\newcommand{\SpecialCharTok}[1]{\textcolor[rgb]{0.37,0.37,0.37}{#1}}
\newcommand{\SpecialStringTok}[1]{\textcolor[rgb]{0.13,0.47,0.30}{#1}}
\newcommand{\StringTok}[1]{\textcolor[rgb]{0.13,0.47,0.30}{#1}}
\newcommand{\VariableTok}[1]{\textcolor[rgb]{0.07,0.07,0.07}{#1}}
\newcommand{\VerbatimStringTok}[1]{\textcolor[rgb]{0.13,0.47,0.30}{#1}}
\newcommand{\WarningTok}[1]{\textcolor[rgb]{0.37,0.37,0.37}{\textit{#1}}}

\providecommand{\tightlist}{%
  \setlength{\itemsep}{0pt}\setlength{\parskip}{0pt}}\usepackage{longtable,booktabs,array}
\usepackage{calc} % for calculating minipage widths
% Correct order of tables after \paragraph or \subparagraph
\usepackage{etoolbox}
\makeatletter
\patchcmd\longtable{\par}{\if@noskipsec\mbox{}\fi\par}{}{}
\makeatother
% Allow footnotes in longtable head/foot
\IfFileExists{footnotehyper.sty}{\usepackage{footnotehyper}}{\usepackage{footnote}}
\makesavenoteenv{longtable}
\usepackage{graphicx}
\makeatletter
\def\maxwidth{\ifdim\Gin@nat@width>\linewidth\linewidth\else\Gin@nat@width\fi}
\def\maxheight{\ifdim\Gin@nat@height>\textheight\textheight\else\Gin@nat@height\fi}
\makeatother
% Scale images if necessary, so that they will not overflow the page
% margins by default, and it is still possible to overwrite the defaults
% using explicit options in \includegraphics[width, height, ...]{}
\setkeys{Gin}{width=\maxwidth,height=\maxheight,keepaspectratio}
% Set default figure placement to htbp
\makeatletter
\def\fps@figure{htbp}
\makeatother
\newlength{\cslhangindent}
\setlength{\cslhangindent}{1.5em}
\newlength{\csllabelwidth}
\setlength{\csllabelwidth}{3em}
\newlength{\cslentryspacingunit} % times entry-spacing
\setlength{\cslentryspacingunit}{\parskip}
\newenvironment{CSLReferences}[2] % #1 hanging-ident, #2 entry spacing
 {% don't indent paragraphs
  \setlength{\parindent}{0pt}
  % turn on hanging indent if param 1 is 1
  \ifodd #1
  \let\oldpar\par
  \def\par{\hangindent=\cslhangindent\oldpar}
  \fi
  % set entry spacing
  \setlength{\parskip}{#2\cslentryspacingunit}
 }%
 {}
\usepackage{calc}
\newcommand{\CSLBlock}[1]{#1\hfill\break}
\newcommand{\CSLLeftMargin}[1]{\parbox[t]{\csllabelwidth}{#1}}
\newcommand{\CSLRightInline}[1]{\parbox[t]{\linewidth - \csllabelwidth}{#1}\break}
\newcommand{\CSLIndent}[1]{\hspace{\cslhangindent}#1}

\KOMAoption{captions}{tableheading}
\makeatletter
\@ifpackageloaded{tcolorbox}{}{\usepackage[skins,breakable]{tcolorbox}}
\@ifpackageloaded{fontawesome5}{}{\usepackage{fontawesome5}}
\definecolor{quarto-callout-color}{HTML}{909090}
\definecolor{quarto-callout-note-color}{HTML}{0758E5}
\definecolor{quarto-callout-important-color}{HTML}{CC1914}
\definecolor{quarto-callout-warning-color}{HTML}{EB9113}
\definecolor{quarto-callout-tip-color}{HTML}{00A047}
\definecolor{quarto-callout-caution-color}{HTML}{FC5300}
\definecolor{quarto-callout-color-frame}{HTML}{acacac}
\definecolor{quarto-callout-note-color-frame}{HTML}{4582ec}
\definecolor{quarto-callout-important-color-frame}{HTML}{d9534f}
\definecolor{quarto-callout-warning-color-frame}{HTML}{f0ad4e}
\definecolor{quarto-callout-tip-color-frame}{HTML}{02b875}
\definecolor{quarto-callout-caution-color-frame}{HTML}{fd7e14}
\makeatother
\makeatletter
\makeatother
\makeatletter
\@ifpackageloaded{bookmark}{}{\usepackage{bookmark}}
\makeatother
\makeatletter
\@ifpackageloaded{caption}{}{\usepackage{caption}}
\AtBeginDocument{%
\ifdefined\contentsname
  \renewcommand*\contentsname{Tabla de contenidos}
\else
  \newcommand\contentsname{Tabla de contenidos}
\fi
\ifdefined\listfigurename
  \renewcommand*\listfigurename{Listado de Figuras}
\else
  \newcommand\listfigurename{Listado de Figuras}
\fi
\ifdefined\listtablename
  \renewcommand*\listtablename{Listado de Tablas}
\else
  \newcommand\listtablename{Listado de Tablas}
\fi
\ifdefined\figurename
  \renewcommand*\figurename{Figura}
\else
  \newcommand\figurename{Figura}
\fi
\ifdefined\tablename
  \renewcommand*\tablename{Tabla}
\else
  \newcommand\tablename{Tabla}
\fi
}
\@ifpackageloaded{float}{}{\usepackage{float}}
\floatstyle{ruled}
\@ifundefined{c@chapter}{\newfloat{codelisting}{h}{lop}}{\newfloat{codelisting}{h}{lop}[chapter]}
\floatname{codelisting}{Listado}
\newcommand*\listoflistings{\listof{codelisting}{Listado de Listados}}
\makeatother
\makeatletter
\@ifpackageloaded{caption}{}{\usepackage{caption}}
\@ifpackageloaded{subcaption}{}{\usepackage{subcaption}}
\makeatother
\makeatletter
\@ifpackageloaded{tcolorbox}{}{\usepackage[skins,breakable]{tcolorbox}}
\makeatother
\makeatletter
\@ifundefined{shadecolor}{\definecolor{shadecolor}{rgb}{.97, .97, .97}}
\makeatother
\makeatletter
\makeatother
\makeatletter
\makeatother
\ifLuaTeX
\usepackage[bidi=basic]{babel}
\else
\usepackage[bidi=default]{babel}
\fi
\babelprovide[main,import]{spanish}
% get rid of language-specific shorthands (see #6817):
\let\LanguageShortHands\languageshorthands
\def\languageshorthands#1{}
\ifLuaTeX
  \usepackage{selnolig}  % disable illegal ligatures
\fi
\IfFileExists{bookmark.sty}{\usepackage{bookmark}}{\usepackage{hyperref}}
\IfFileExists{xurl.sty}{\usepackage{xurl}}{} % add URL line breaks if available
\urlstyle{same} % disable monospaced font for URLs
\hypersetup{
  pdftitle={Mapeo de Erosión del Suelo a Nivel Nacional empleando la Metodología de RUSLE y rgee 📚🏞️🌱🌧️🚜🔎📉💨🌾🗺️ },
  pdfauthor={ Antony Barja  Revizado por: Ing.Carlos Carbajal  Afiliación: INIA-LABSAF},
  pdflang={es},
  colorlinks=true,
  linkcolor={blue},
  filecolor={Maroon},
  citecolor={Blue},
  urlcolor={Blue},
  pdfcreator={LaTeX via pandoc}}

\title{Mapeo de Erosión del Suelo a Nivel Nacional empleando la
Metodología de RUSLE y rgee📚🏞️🌱🌧️🚜🔎📉💨🌾🗺️}
\author{ Antony Barja\\
Revizado por: Ing.Carlos Carbajal\\
Afiliación: INIA-LABSAF}
\date{2023-08-02}

\begin{document}
\maketitle
\ifdefined\Shaded\renewenvironment{Shaded}{\begin{tcolorbox}[enhanced, sharp corners, interior hidden, borderline west={3pt}{0pt}{shadecolor}, boxrule=0pt, frame hidden, breakable]}{\end{tcolorbox}}\fi

\renewcommand*\contentsname{Tabla de contenidos}
{
\hypersetup{linkcolor=}
\setcounter{tocdepth}{2}
\tableofcontents
}
\bookmarksetup{startatroot}

\hypertarget{bienvenids}{%
\chapter*{!Bienvenid@s! 👋}\label{bienvenids}}
\addcontentsline{toc}{chapter}{!Bienvenid@s! 👋}

\markboth{!Bienvenid@s! 👋}{!Bienvenid@s! 👋}

Este manual lleva por título \emph{``Mapeo de Erosión del Suelo a Nivel
Nacional empleando la Metodología RUSLE y rgee''}. \textbf{El objetivo,
de esta guía dar a conocer a los usuarios y nuevos usuarios un flujo de
trabajo reproducible y replicable para cualquier área de estudio que
desea estimar y analizar los patrones espaciales y temporales de la
erosión de suelo}.

Bajo el conexto de la ciencia datos y el apogeo de la mineria de datos
(\textbf{big data} o \textbf{data mining}), esta guía dará a conocer el
fácil acceso y procesamientos automatico de datos geográficos
(\textbf{geocomputación o geografía computacional}) que permitan su
manipulación para el cálculo de la erosión del suelo sin tener la
necesidad de contar con elevadas capacidades computacionales.

Finalmente, para poder reproducir y replicar satisfactoriamente este
manual se debe tener cierto acercamiento a cualquier lenguaje de
programación orientado a objetos (\textbf{R},\textbf{python},
\textbf{JavaScript}, \textbf{etc}), asimismo se dará ciertas
requerimientos que deberá presentar tu laptop o pc para la instalación
de software con el fin de cumplir los objetivos mostrados inicialmente,
es necesario recalcar que todo los procesos que se darán a conocer en
los próximos cápitulos serán desarrollado sólo bajo el lenguaje de
programación de \textbf{R}.

\bookmarksetup{startatroot}

\hypertarget{aspectos-generales-e-instalaciuxf3n-de-software}{%
\chapter{Aspectos generales e instalación de
software}\label{aspectos-generales-e-instalaciuxf3n-de-software}}

En está sección se mostrará los conceptos básicos del software R, Rtools
y Rstudio como tambien la definición de los paquetes que se utilizara en
el flujo de trabajo.

\begin{tcolorbox}[enhanced jigsaw, opacityback=0, toprule=.15mm, colback=white, breakable, arc=.35mm, colframe=quarto-callout-note-color-frame, leftrule=.75mm, rightrule=.15mm, left=2mm, bottomrule=.15mm]

\textbf{Atención:}\vspace{2mm}

Para poder replicar y reproducir este manual sin ningun problema es
necesario tener como mínimo las siguientes características
computacionales:

\begin{itemize}
\tightlist
\item
  Memoria RAM : 4-8 GB
\item
  Capacidad de almacenamiento: 255GB
\item
  Procesador como mínimo: i7 con CPUde 2.40 GHz
\end{itemize}

\end{tcolorbox}

\hypertarget{quuxe9-es-r}{%
\section{¿Qué es R?}\label{quuxe9-es-r}}

Es un lenguaje de programación interpretado de código abierto
multi-plataforma que permite hacer diferentes tipos de análisis
estadísticos, desde importar datos, ordenarlos, modelar y visualizar
mediante gráficos de alta calidad, e incluir en informes académicos de
manera científica (\textbf{Hadley Wickham y Garrett Grolemund,2017} ).

\includegraphics{index_files/mediabag/r_native.png}

\textbf{Características:}

\begin{itemize}
\tightlist
\item
  Es un software libre y de código abierto.
\item
  Corre en multiples sistemas operativos (GNU/Linux, MacOSX y Windows).
\item
  Posee una variedad de paquetes para temás específicos.
\item
  Comunidad científica muy dinámica.
\end{itemize}

\hypertarget{quuxe9-es-rtools}{%
\section{¿Qué es Rtools?}\label{quuxe9-es-rtools}}

Es una colección de herramientas necesarias en R para compilar y
construir paquetes desde el código fuente en solo los sistemas
operativos de \textbf{Windows}, si no se tiene instalado, no se podrían
instalar ni utilizar muchos paquetes que requieren está compilación
basadas en C o C++.

\hypertarget{quuxe9-es-rstudio}{%
\section{¿Qué es Rstudio?}\label{quuxe9-es-rstudio}}

Es un entorno de desarrollo integrado (IDE) para R. Incluye una consola,
un editor de código, una consola, un gestor para la administración del
espacio de trabajo, entre otros.

CONCLUSIÓN: ``RSTUDIO ES EL ROSTRO BONITO DE R''

\includegraphics{index_files/mediabag/parts.png}

\hypertarget{instalaciuxf3n-de-r-rtools-y-rstudio}{%
\section{Instalación de R, Rtools y
Rstudio}\label{instalaciuxf3n-de-r-rtools-y-rstudio}}

\url{https://www.youtube.com/embed/h2IPWVXaUuU}

\hypertarget{cuuxe1l-es-el-concepto-de-un-paquete-en-r}{%
\section{¿Cuál es el concepto de un paquete en
R?}\label{cuuxe1l-es-el-concepto-de-un-paquete-en-r}}

Un paquete en R es una colección organizada de funciones, datos y
documentación que se agrupa para resolver un conjunto particular de
problemas (\textbf{Hadley Wickham, 2021}).

Estos son los paquetes que se utilizará durante todo flujo de trabajo 👇

\hypertarget{instalaciuxf3n-de-paquetes-en-r}{%
\section{Instalación de paquetes en
R}\label{instalaciuxf3n-de-paquetes-en-r}}

Una vez instalada los software de R y Rstudio procederemos a instalar y
configurar todos los paquetes mencionados en la sección de
\textbf{requerimientos}.

Este proceso solo se realizar una única vez, cuando ya tienes instalado
y configurado todos los paquetes solo es necesario activarlos en tu
sistema.

\begin{Shaded}
\begin{Highlighting}[]
\CommentTok{\# Lista de paquetes a instalar}
\NormalTok{pkgs }\OtherTok{\textless{}{-}} \FunctionTok{c}\NormalTok{(}\StringTok{"tidyverse"}\NormalTok{, }\StringTok{"tidyterra"}\NormalTok{,}\StringTok{"rgee"}\NormalTok{ ,}\StringTok{"tmap"}\NormalTok{, }\StringTok{"gifski"}\NormalTok{, }\StringTok{"cloudml"}\NormalTok{, }\StringTok{"sf"}\NormalTok{)}
\CommentTok{\# install.packages(pkgs = pkgs, dependencies = TRUE)}
\CommentTok{\# Se recomienda instalar la versión de GitHub de rgee}
\end{Highlighting}
\end{Shaded}

Para verificar que todo los paquetes fuerón instalados, corremos el
siguiente script:

\begin{Shaded}
\begin{Highlighting}[]
\CommentTok{\# Verificar la instalación de los paquetes}
\ControlFlowTok{for}\NormalTok{ (pkg }\ControlFlowTok{in}\NormalTok{ pkgs) \{}
  \ControlFlowTok{if}\NormalTok{ (}\SpecialCharTok{!}\FunctionTok{requireNamespace}\NormalTok{(pkg, }\AttributeTok{quietly =} \ConstantTok{TRUE}\NormalTok{)) \{}
    \FunctionTok{message}\NormalTok{(}\FunctionTok{paste}\NormalTok{(}\StringTok{"El paquete"}\NormalTok{, pkg, }\StringTok{"no está instalado."}\NormalTok{))}
\NormalTok{  \} }\ControlFlowTok{else}\NormalTok{ \{}
    \FunctionTok{message}\NormalTok{(}\FunctionTok{paste}\NormalTok{(}\StringTok{"El paquete"}\NormalTok{, pkg, }\StringTok{"está instalado y cargado correctamente."}\NormalTok{))}
\NormalTok{  \}}
\NormalTok{\}}
\end{Highlighting}
\end{Shaded}

\begin{verbatim}
El paquete tidyverse está instalado y cargado correctamente.
El paquete tidyterra está instalado y cargado correctamente.
El paquete rgee está instalado y cargado correctamente.
El paquete tmap está instalado y cargado correctamente.
El paquete gifski está instalado y cargado correctamente.
El paquete cloudml está instalado y cargado correctamente.
El paquete sf está instalado y cargado correctamente
\end{verbatim}

\bookmarksetup{startatroot}

\hypertarget{registro-de-usuario-en-la-plataforma-de-google-earth-engine}{%
\chapter{Registro de usuario en la plataforma de Google Earth
Engine}\label{registro-de-usuario-en-la-plataforma-de-google-earth-engine}}

\hypertarget{registro-de-gmail}{%
\section{Registro de GMAIL}\label{registro-de-gmail}}

Para poder acceder a todos los beneficios que nos ofrece la plataforma
de Earth Engine es necesario contar con una cuenta de GMAIL activo.

Para registrarnos nos dirigimos a 👉 :
\url{https://code.earthengine.google.com/register}

\includegraphics{https://user-images.githubusercontent.com/23284899/258827533-6b7e7038-7134-4ced-ac30-ed13bf17cd33.png}

Earth Engine te ofrece dos alternativas para poder registrarte:

\begin{itemize}
\tightlist
\item
  \href{https://earthengine.google.com/commercial/}{Usuario comercial}:
  Desarrollo de productos comerciales,monetization de servicos, etc.
\item
  \href{https://earthengine.google.com/noncommercial/}{Usurio no
  comercial}: Investigación, docencia, etc.
\end{itemize}

Para nuestro interés procedemos a elegir la segunda opción.

\includegraphics{https://user-images.githubusercontent.com/23284899/258835768-d84c50b2-0c74-4ee9-a7d8-c5bf7137360b.png}

Este nos llevará a una nueva penstaña en donde nos solicitará llenar
nuestros datos personales y algunas preguntas adicionales como la
afiliación y cuales son las intenciones de usar Earth Engine.

\includegraphics{https://user-images.githubusercontent.com/23284899/258846389-57859ce1-19df-44f7-924f-0958591ef78e.png}

Finalizado el registro Earth Engine, solo nos queda esperar el correo de
confirmación para poder acceder sin ninguna restricción a la plataforma.
Es necesario tener en cuenta que la habilitación puede tomar un lapso de
tiempo, hoy en día es casi de forma automática, pero esto podría tomar
en algunos casos entre 1 a 2 días.

\includegraphics{https://user-images.githubusercontent.com/23284899/258846473-981e1659-82cc-477c-93c2-bbf1cf89ee1f.png}

\hypertarget{confirmaciuxf3n-de-earth-engine}{%
\section{Confirmación de Earth
Engine}\label{confirmaciuxf3n-de-earth-engine}}

Con este correo de bienvenida de Earth Engine, podemos estár al 100\%
seguro de tener acceso a :

\begin{itemize}
\tightlist
\item
  Earth Engine Code Editor
\item
  Earth Engine Developer Docs
\item
  Eartg Engine Explore
\end{itemize}

\includegraphics{https://user-images.githubusercontent.com/23284899/258848622-c362d3ca-bb81-419c-b273-2fa4be01f44f.png}

\begin{tcolorbox}[enhanced jigsaw, opacityback=0, toprule=.15mm, colback=white, breakable, arc=.35mm, colframe=quarto-callout-caution-color-frame, leftrule=.75mm, rightrule=.15mm, left=2mm, bottomrule=.15mm]

\textbf{Observacón:}\vspace{2mm}

La vinculación de GMAIL hasta el correo de configuración por parte de
Earth Engine, es la base fundamental para posteriormente trabajar en
cualquier lenguage de programación que consume la API de Earth Engine
(R, python, Js, Julia).

\end{tcolorbox}

\hypertarget{configuraciuxf3n-de-gcloudcli}{%
\section{Configuración de
GCloudCLI}\label{configuraciuxf3n-de-gcloudcli}}

Para poder vincular Google Earth Engine con R, es necesario contalar con
la instalación de GCloud CLI, para instalar este software desde R solo
necesitamos cargar la librería \texttt{cloudml} y utilizar el siguiente
comando \texttt{gcloud\_install()}.

\begin{Shaded}
\begin{Highlighting}[]
\CommentTok{\# Activación o llamado de los paquetes instalados}
\FunctionTok{library}\NormalTok{(cloudml)}
\CommentTok{\# Instalación de GCloud CLI}
\FunctionTok{gcloud\_install}\NormalTok{()}
\end{Highlighting}
\end{Shaded}

\hypertarget{registro-de-crendenciales}{%
\section{Registro de crendenciales}\label{registro-de-crendenciales}}

\hypertarget{configuraciuxf3n-de-rgee}{%
\section{Configuración de rgee}\label{configuraciuxf3n-de-rgee}}

\hypertarget{hola-mundo-en-rgee}{%
\section{Hola mundo en rgee}\label{hola-mundo-en-rgee}}

\bookmarksetup{startatroot}

\hypertarget{quuxe9-es-rusle}{%
\chapter{¿Qué es RUSLE?}\label{quuxe9-es-rusle}}

Metodología desarrollada por Renard et al (1991) que permite estimar la
perdida media anual de suelo en función a un modelo matemático.

\begin{equation}
    A = R \cdot K \cdot LS \cdot C \cdot P 
\end{equation}

Donde los cinco parámetros de entrada están relacionados con la
precipitación, las características del suelo, la topografía, el manejo
de la cubierta y los cultivos y las prácticas de conservación.

\hypertarget{definiciuxf3n-de-paruxe1metros}{%
\section{Definición de
parámetros}\label{definiciuxf3n-de-paruxe1metros}}

\hypertarget{factor-r}{%
\subsection{Factor R 🌧️}\label{factor-r}}

Es el factor de erosividad de escorrentía de lluvia
(MJ·mm·ha−1·h−1·Año−1), índice numérico que expresa la capacidad de la
lluvia para erosionar el suelo. Para su cálculo existen diferentes
modelos; sin embargo para este manual se consideró usar la formula de
Wischmeier y Smith (1978) presentada el paper de \href{}{Zubairul Isla}
la cual la base de referencia.

\begin{equation}
    R = 1.73 \times 10^{(1.5\times\log(Pm^2/Pa)-0.08188)}   
\end{equation}

Donde:

\begin{itemize}
\tightlist
\item
  R es la erosividad de lluvia en \((MJ\) en
  \(mm)/(ha^{-1} h^{-1} año^{-1})\) (mega julios por milímetro por
  hectárea por hora por año)
\item
  Pm es la precipitación mensual
\item
  Pa es la precipitación en un año
\end{itemize}

\hypertarget{factor-k}{%
\subsection{Factor K}\label{factor-k}}

Es el factor de erodibilidad del suelo \((Mg·h·MJ^{−1}·mm^{−1})\), una
descripción numérica de la susceptibilidad de las partículas del suelo a
la erosión hídrica. Estos valores van de 0 a 1, donde 0 es menos
suceptible y 1 es altamente susceptible a la erosión hídrica.

Para su cálculo se tomo como referencia la formula de Sharpley and
Williams (1990).

\begin{equation}
    K = [0.2 + 0.3 \times exp(-0.0256 \times SAN \times (1 - \frac{SIL}{100}))]\times[1-\frac{0.25\times CLA}{CLA + exp(3.72 - 2.95\times CLA)}]  
\end{equation}

Donde:

\begin{itemize}
\item
  SAN: Porcentaje de arena
\item
  SIL: Porcentaje de limo
\item
  CLA: Porcentaje de arcilla
\item
  SN: 1 - SAN/100
\end{itemize}

\hypertarget{factor-ls}{%
\subsection{Factor LS}\label{factor-ls}}

Es el factor de longitud de pendiente (L) y el factor de inclinación
(S), ambos variables combinadas expresan el efecto de la topografía
local sobre la tasa dde erosión del suelo. Para su cálculo se tomo en
cuenta la formula establecida por Moore (1985).

\begin{equation}
    LS = (0.4 + 1) \times (Flowacc\times CellSize /22.13)^{0.4} \times (\sin(\theta)/0.0896)^{1.3} 
\end{equation}

Donde:

\begin{itemize}
\item
  Flowacc : Acumulación de flujo
\item
  CellSize: Tamaño de pixel
\item
  \(\theta\) : Mapa de pendientes en grados
\end{itemize}

\hypertarget{factor-c}{%
\subsection{Factor C}\label{factor-c}}

Determina la eficacia relativa de los sistemas de manejo del suelo y de
los cultivos en terminos de prevencion o reduccion de la perdida de
suelo. Los valores oscila entre 0, para una superficie no erosible, y 1,
parcela desnuda (sin vegetación). Por temas práticos muchos autores
adoptan enfoques simplificados: por ejemplo, utilizar el mapas de
cobertura del suelo y asignando un factor C a cada clase, o por ultima
emplear los valores de NDVI y las condiciones climáticas.

Para su cálculo se tomo como referencia la formula de Almagro et al
(2019).

\begin{equation}
    C = 0.1 \times(\frac{-NDVI + 1}{2})
\end{equation}

Donde:

\begin{itemize}
\item
  Flowacc : Acumulación de flujo
\item
  CellSize: Tamaño de pixel
\item
  \(\theta\) : Mapa de pendientes en grados
\end{itemize}

\hypertarget{factor-p}{%
\subsection{Factor P}\label{factor-p}}

Recoge la influencia que tienen las prácticas de conservación de suelos
sobre las tasas de erosión, en otras palabras el factor P tiene como
objetivo reducir la escorrentía del agua y, en consecuencia, la pérdida
de suelo.

Para su cálculo se tomo en consideración la tabla de valores propuestos
por , donde se establece lo siguienente:

\begin{longtable}[]{@{}ll@{}}
\toprule\noalign{}
Clasificación de Usos de Suelo & P \\
\midrule\noalign{}
\endhead
\bottomrule\noalign{}
\endlastfoot
Bosque & 0.8 \\
Tierras de cultivo & 0.5 \\
Construcción & 1.0 \\
Vegetación esparcida & 1.0 \\
Cuerpos de agua & 1.0 \\
Matorral & 1.0 \\
Humedales & 1.0 \\
\end{longtable}

https://www.frontiersin.org/articles/10.3389/fenvs.2023.1136243/full

\bookmarksetup{startatroot}

\hypertarget{rusle-con-rgee}{%
\chapter{RUSLE con rgee}\label{rusle-con-rgee}}

\bookmarksetup{startatroot}

\hypertarget{resultados}{%
\chapter{Resultados}\label{resultados}}

\bookmarksetup{startatroot}

\hypertarget{graficos-comparativos}{%
\chapter{Graficos comparativos}\label{graficos-comparativos}}

\bookmarksetup{startatroot}

\hypertarget{mapas}{%
\chapter{Mapas}\label{mapas}}

\bookmarksetup{startatroot}

\hypertarget{animaciuxf3n}{%
\chapter{Animación}\label{animaciuxf3n}}

\bookmarksetup{startatroot}

\hypertarget{referencias}{%
\chapter*{Referencias}\label{referencias}}
\addcontentsline{toc}{chapter}{Referencias}

\markboth{Referencias}{Referencias}

\hypertarget{refs}{}
\begin{CSLReferences}{0}{0}
\end{CSLReferences}



\end{document}
